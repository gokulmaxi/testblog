\documentclass{article}
\usepackage{hyperref}
\title{Setup blog with Github Actions}
\author{Michal Hoftich}
\begin{document}
\maketitle

Git doesn't keep modification time in the repository. As we want to compile only modified
\LaTeX\ files, we need to restore the modification times. 
Thankfully, this is possible thanks to 
\href{https://github.com/chetan/git-restore-mtime-action}{git-restore-mtime action}
which can restore modification times from the file commit times.

For the correct function, it needs the following instruction in the workflow file:

\begin{verbatim}
- uses: actions/checkout@v2
  with:
    fetch-depth: 0
\end{verbatim}

\href{https://github.com/stefanzweifel/git-auto-commit-action}{git-auto-commit Action}

It is possible to request compilation on the Github server without need to run
make4ht locally first. You just need to create the .published file and push it to Github.

The file can be created using the following command:

\begin{verbatim}
$ date +%s > tex_filename.published
\end{verbatim}

\end{document}
