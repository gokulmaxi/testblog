\documentclass{article}
\title{Jekyll Setup}
\author{Michal Hoftich}
\def\makeht{\texttt{make4ht}}
\usepackage{hyperref}
\begin{document}
\maketitle

I've showed how to use \makeht\ \texttt{staticsite} extension
in the \href{/testblog/2021/07/30/how-to-blog-with-tex4ht.html}
{previous article}. In this article, I will show how to 
setup it together with \href{https://jekyllrb.com/}{Jekyll}
to create a simple blog.




\begin{verbatim}
$ jekyll new docs
\end{verbatim}

\begin{verbatim}
show_excerpts: true
excerpt_separator: "</p>"
\end{verbatim}


\begin{verbatim}
docs/_includes/head.html
\end{verbatim}

\begin{verbatim}
<head>
  <meta charset="utf-8">
  <meta http-equiv="X-UA-Compatible" content="IE=edge">
  <meta name="viewport" content="width=device-width, initial-scale=1">
  
  <link rel="stylesheet" href="{{ "/assets/main.css" | relative_url }}" />
  
  <link rel="stylesheet" href="/css/{{style}}" />
  
  
  
    
  
</head>
\end{verbatim}

\end{document}
