\documentclass{article}
\usepackage{hyperref}
% \usepackage{markdown}
\title{Blogging in LaTeX, 2021 edition}
\author{Michal Hoftich}
\begin{document}
\maketitle

% \begin{markdown*}{hybrid=true}

This is first in series of posts about blogging with \LaTeX\ using
\href{https://tug.org/tex4ht/}{\TeX4ht}. I've set it up as a showcase of
\TeX4ht features for \href{https://tug.org/tug2021/}{TUG 2021} conference.
Due to timing constraints, I haven't show it in the end, but I think it 
can be interesting to some people anyway.

\section*{Why would you do such a crazy thing?}



\section*{Overview of the setup}

\begin{enumerate}
\item Use \href{/testblog/2021/07/30/how-to-blog-with-tex4ht.html}
{\TeX4ht to produce files suitable for static site generators}.
\item Use static site generator, like \href{https://jekyllrb.com/}{Jekyll}
or \href{https://gohugo.io/}{Hugo}
, to produce a web site.
\item Use Github Actions to automatically compile \LaTeX\ files pushed
to the repository, and recompile the web site. 
\end{enumerate}

I will tell more on how to set up an automatic
blog publishing using Github Pages and Github Actions in the following posts.
% \end{markdown*}

\end{document}
