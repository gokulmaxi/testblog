\documentclass{article}
\usepackage{hyperref}
\title{How to Blog With TeX4ht}
\author{Michal Hoftich}
\def\makeht{\texttt{make4ht}}
\begin{document}
\maketitle

This post is a part of series on how to set up
\href{https://tug.org/tex4ht/}{TeX4ht}, \LaTeX\ to XML converter
for use with Static Site Generators. We will discuss
how to configure it to produce suitable HTML in this article.

\section{Static site extension for make4ht}

Conversion process used by \TeX4ht is quite complex. It needs to 
compile \LaTeX\ file to a DVI file with special instructions 
inserted by \texttt{tex4ht.sty} package. This DVI file is then 
processed by \texttt{tex4ht} command, that produces HTML or XML files,
and instructions for the last command, \texttt{t4ht}, which produces
CSS files and pictures. Traditionally, this process was handled 
by the \texttt{htlatex} script, but it had many weaknesses, so 
the currently recommended build tool is \makeht. 
You can find some details about \texttt{htlatex} and \makeht\ 
differences in the 
\href{https://www.kodymirus.cz/make4ht/make4ht-doc.html#difference-of-texttt-makeht}
{\makeht\ documentation}.

Among features \makeht\ has, are Lua build files, post-processing filters, and extension support. 
Filters can clean-up the generated files, and fix some common issues that 
are hard to fix on the \TeX\ level. They can be required either from 
the Lua build files, or using extensions. 

\makeht\ provides an extension that aims at support for static site generators. We can show the usage
on a simple example:

\begin{verbatim}

\documentclass{article}
\begin{document}
\title{Hello world test}
\author{Michal}
\maketitle

This is my test post.
\end{document}
\end{verbatim}

You can use a following command to generate file suitable for static site generators:

\begin{verbatim}
make4ht -f html5+staticsite filename.tex
\end{verbatim}

By default, \texttt{staticsite} extension produces file named as \verb|YYYY-MM-DD-<filename>|,
this example can be named as \texttt{2021-07-25-filename.html}. It is not 
ordinary HTML file, but it contains YAML header with document metadata:

\begin{verbatim}
---
meta:
- charset: 'utf-8'
- name: 'generator'
  content: 'TeX4ht (https://tug.org/tex4ht/)'
- name: 'viewport'
  content: 'width=device-width,initial-scale=1'
- name: 'src'
  content: '2021-07-18-hello-world.tex'
time: 1626619562
updated: 1627244699
styles:
- '2021-07-18-hello-world.css'
title: 'Hello world test'
---

   
<!--  l. 7  --><p class='indent'>   This is my test post.
</p>
    
\end{verbatim}

Although most static site generators expects Markdown, they also accept HTML files in this form.
When \texttt{staticsite} is used for the first time, it creates file with a \texttt{.published}
extension. It contains timestamp of the moment when it was used for the first time. This 
timestamp is used for the date part of the generated filename.

\section{Copy the generated files to the static site}

The \texttt{staticsite} extension can copy the generated files to places where the static site
generator expects files to process.

Let's say, that we have the following directory structure:

\begin{verbatim}
blog/
.. texposts_root/
.... first_post/
...... first_post.tex
...... first_post.published
.... second_post/
...... second_post.tex
...... second_post.published
.. html_root/
.... 2021-07-18-first-post.html
.... 2021-07-19-second-post.html
.. .make4ht
\end{verbatim}


The blog's main directory contains file \texttt{.make4ht}, and two directories: \verb|texposts_root|
and \verb|html_root|. 

The source \LaTeX\ files are stored in subdirectories of \verb|texposts_root|. We want to copy the
generated HTML files to \verb|html_root| automatically. The \texttt{staticsite} extension can be 
configured to do that using the \texttt{.make4ht} configuration file. This file is meant for passing
of shared configuration to \makeht, like in this case, where we want to specify to copy all generated
files to the \verb|html_root| directory.




\end{document}
